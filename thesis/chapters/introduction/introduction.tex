\chapter*{Introduction}
\addcontentsline{toc}{chapter}{Introduction}

\defcitealias{MaResFut20}{MaResFut20}
\defcitealias{Applit21}{Applit21}
\defcitealias{Chck21}{Chck21}

In the past few years, the web scraping and data extraction industry became much more prominent, as the need for data rises among all branches of science. 
The industry is expected to grow at $13.1\%$ CAGR, reaching a market value of USD $948.60$ Million \citepalias{MaResFut20}.
\par
With web browser automation also being the leading technology for UI testing and \ac{RPA} for web, the technology exceeds the data extraction needs by far.
Despite the immense size of this evergrowing industry, there is still no standardized universal format for storing automated workflows. 
Most automation developers produce executable code in general purpose programming languages, with \textit{Java}, \textit{JavaScript}, and \textit{Python} being the most prominent ones \citepalias{Applit21}.
\par
This approach poses a certain security risk, as the user of such automation needs to run untrusted code. 
It also creates a barrier to entry for beginners without the required programming knowledge. 
Furthermore, the absence of a standardized format hinders the collaboration between developers.

\section*{Thesis goals}
The goal of this thesis is to develop a human-readable, declarative format for storing and creating web automations, with an interpreter of this format and a visual editor, allowing less technical users to create and maintain automations in this format.
\par
Such format should allow for development of resilient, reusable, and comprehensive workflow definitions. 
It should also be machine-readable, parsable and editable, ultimately leading to a simpler adoption of the format by developers of third-party software.
\par
This format should also be application-oblivious i.e. not too oriented on automating only web-related workflows.
The definition of the format should allow developers to create interpreters for this format for handling different automation tasks, still maintaining the same syntax.
The presented workflow interpreter should then be able to parse, validate and execute the defined web-related workflows. 
It should also implement a basic programmable interface to allow other developers to use the interpreter from their own software.
\par
Implemented as a web application, the workflow editor should be able to generate valid workflow files, allowing the user to create the workflow definitions without knowing the exact internal syntax of the definition format.
The editor should implement a user-friendly, with an intuitive \ac{GUI} and a steep learning curve.
The ultimate goal of the editor is to shield the user from the programming part of the automation task completely, leaving them with a simple yet powerful graphical tool.

\defcitealias{PPatch21}{PPatch21}
\defcitealias{PReadme22}{PReadme22}

\section*{Related Work}
As of now, there are already numerous solutions for automating web actions on the market. 
A majority of those uses existing web browsers and offer a programmable interface for simulating user input.

\textbf{Cypress} is a \ac{E2E} Javascript testing framework containing various assertions for \ac{QA} testing of webpages.
It supports multiple web browsers and offers its own UI and toolkit for test programming and running. 
Due to its strong orientation towards testing, it does not provide much methods for data extraction and crawling.

\textbf{Selenium WebDriver} is a fairly popular tool among web UI testers, as it offers a wide variety of methods for \ac{QA} testing.
Distributed as a multilanguage library, Selenium implements a high-level interface for controlling web browsers from code.
% Aside from regular commercial web browsers, Selenium also implements interfaces for PhantomJS and HTMLUnit, both headless scriptable browsers used as a lightweight alternative to regular browsers.

\textbf{Puppeteer} is a low-level library used for web browser automation. 
Unlike Cypress and Selenium, Puppeteer supports Chrome (or Chromium) as its only backend browser as of now (\today).

The communication with the browser is implemented via WebSockets and the DevTools Protocol, a Chromium-specific set of commands.
This allows Puppeteer to exceed Selenium both in stability and performance, sporting up to $17\%$ speedup in benchmarks \citepalias{Chck21}.

\textbf{Playwright} is another low-level library multilanguage library offering programmable ways of controlling web browser.
For browser communication, Playwright uses similiar technology as Puppeteer, unlike Puppeteer, Playwright supports multiple commercial browsers (Chromium, Firefox, Webkit as of \today).

Due to differences between browsers and partial incompatibility of protocols, Playwright is distributed with patched versions of Firefox and Webkit \citepalias{PPatch21}.
Stock versions of Chromium based browsers (Google Chrome, Microsoft Edge) are supported. \citepalias{PReadme22}

\vspace{\baselineskip}

All the aforementioned examples however require programming, which can mean a significant barrier to entry for beginners.