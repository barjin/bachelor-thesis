\chapter{Conclusion}

The goal of this work, as stated in the \hyperref[intro]{Introduction} was to \textit{create a human-readable, declarative format for
storing and creating web automations, with an interpreter of this format and a
visual editor, allowing less technical users to create and maintain automations in
this format.}

By implementing the \textit{Editor} application and the \textit{Runner} module, this goal has been achieved in full scale.
Both parts of the project are now fully functional - in accordance with the requirements from the \autoref{requirements} Requiements - and can be used to create and edit web automations
for data extraction and process automation.
Both parts of the project have been developed in a modular fashion, which makes the further development process easier and faster.

Possible future work on the \textit{Editor} application and the \textit{Runner} module is to implement more advanced features.
For instance, less technical users might prefer an automation recorder, implemented as a browser extension.
Such a tool might generate a web automation file in the workflow definition format automatically, without any need for manual editing.
This application might also reuse the components from the \textit{Editor} application to allow for additional workflow editing and inspection.

For more technical users, the debugging features in the \textit{Editor} application could also be further enhanced, as the simplicity of the definition format might allow for more advanced 
analysis features. Such an application might reuse the analysis methods from the \texttt{wbr-interpret} package.