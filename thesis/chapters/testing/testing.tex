\chapter{Testing}

During the development phase, all parts of the project have been tested using various testing methods.
The following chapter describes the testing methodology and the nature of the individual tests.
Aside from testing the software using automated test suites, user testing was carried out on the \textit{Editor} application.

\section{Code quality}

While the code quality does not directly influence the correctness of the results, it is an important aspect of the project, as it influences readability and maintainability of the code.
Maintaining a consistent code style also speeds up the development process and reduces the risk of introducing bugs.

The \texttt{wbr-interpret} module and the \textit{Editor} application are written in \textit{TypeScript} and \textit{TSX}, respectively.
The codebase of the entire project follows the practices described in the \textit{Airbnb JavaScript Style Guide}\footnote{\href{https://airbnb.io/javascript/}{https://airbnb.io/javascript/}}, made and maintained by \textit{Airbnb, Inc.}
\textit{Airbnb Inc.} also provides a \texttt{.eslintrc} file for configuring \textit{ESLint} in accordance with the practices described in the \textit{Style Guide}.
All the code in the project is then automatically checked for conforming to the style guide using \texttt{ESLint} already during the development phase, which simplifies formatting and allows us to see potential errors right away.

The project's GitHub repository also contains an automated \textit{GitHub Actions} workflow, running the \textit{ESLint} check with every push to the repository.
This helps with catching the bad code patterns early on and prevents the need for further manual testing.

\section{Automated tests}

To ensure the elementary correctness of the project parts, the project has contained automated test suites since the early stage of development.
This helps both with code maintenance and new feature implementation, as the automated test suites ensure we do not introduce any breaking changes.

All of the tests mentioned in the following sections are also ran in the corresponding \textit{GitHub Actions} workflow. 
This ensures that the code available in the main branch of the public repository is always tested against the latest changes.

All the following tests were carried out using the \texttt{jest}\footnote{\href{https://jestjs.io/}{https://jestjs.io/}} testing framework, if not stated otherwise.

\subsection{Unit tests}

The entirety of the \texttt{wbr-interpret} package is covered with unit tests testing the individual components of the package.
Special attention is paid to the \textit{Interpreter} and \textit{Preprocessor} classes, as those are the core of the project
and contain the largest amount of complex code. 

The success of the unit tests is directly related to the code design of the module, 
since almost all methods in both classes were first written as pure functions, and only then they were converted to classes for
encapsulation and to provide a more convenient interface.

\smallskip
While it is possible to unit test components in a React application, most of the components in the \textit{Editor} application are trivial.
For this reason, there are no unit tests for the \textit{Editor} application and all the testing is carried out by \ac{E2E} tests.

\subsection{E2E tests}

Both the \textit{Editor} application and \texttt{wbr-interpret} package have been tested with \acl{E2E} tests as well.
These tests are typically longer and follow a complete user path, testing the entire application from the user's perspective.

The \texttt{wbr-interpret} package is \acl{E2E} tested for two different scenarios:
\begin{itemize}
    \item Loading, validating and executing a simple workflow.
    \item Loading, validating, initializing and executing an advanced, parametrized workflow.
\end{itemize}

Both \ac{E2E} tests are run against a local server, which is started before the tests are run.
This way we can observe the actions carried out by the \textit{Runner} even without obtaining any output data.

The \textit{Editor} application has been tested using \textit{Playwright} library to simulate the user's interaction with the application.
There are two tested scenarios:
\begin{itemize}
    \item Uploading various workflow files and seeing which one is valid. 
    \item Creating new automation, editing it, running it, seeing the results and downloading the workflow definition file.
\end{itemize}

All the automations are also run against a local server, which allows us to control the content of the webpages 
and observe the actions caused by the automation execution.

\section{User testing}

Aside from the code testing, the \textit{Editor} application has also been tested by 8 potential users of different technical backgrounds.
This was done in two parts.
First, the users were asked to use the \textit{Editor} application to create a simple automation file.
After getting accustomed to the software, the users were given a \ac{SUS} survey and were asked to rate the application's usability.

\subsection{Test scenario}

The task for the users testing the \textit{Editor} application was to create a simple web automation using the application and execute it to validate the functionality.
Such a task requires utilizing a number of \acs{UI} elements of the \textit{Editor} application, while not being too demanding or requiring much technical knowledge.

\smallskip
The created automation should be able to perform the following steps:
\begin{enumerate}
    \item Navigate to \href{https://jindrich.bar/}{https://jindrich.bar/}.
    \item Click an arbitrary link on the page.
    \item If the resulting page contains paragraphs with text, download these first, otherwise just terminate the execution.
\end{enumerate}
    
The users were provided no additional support during the task.
Out of the eight users, only three of them were able to successfully complete the task, while the others failed at various points.
The successful users were all computer specialists with a strong technical background.
While this might show a trend towards excessive technicality of the application, the sample size is not significant enough to make any conclusions.

The individual user complaints bring better insight into the why some users failed to complete the task.
The less successful users complained mostly about the complexity of the workflow definition format and
the confusing logic of the workflow interpretation.

There were no direct complaints about the application's user interface layout. 
After concluding the experiment, all of the users were instinctively able to find all the control elements when asked to perform a specific action.
All of them were also able to answer questions about the system state.

\subsection{SUS survey}
All the testers have also shared their opinions on the usability of the application via the \ac{SUS} survey.
The detailed results of the \ac{SUS} survey are included in the \hyperref[sec:sussurvey]{Attachment A.1} SUS Survey Results.
The attachment also contains additional information about the survey itself.

The average \ac{SUS} score over the ratings of the eight users is \textbf{52.5}.
This sets the \textit{Editor} application somewhere between the \textit{10th} and \textit{20th} percentile of typical \ac{SUS} evaluated systems.
While this might come off as negative result, it is still a valuable indicator of the quality of the \textit{Editor} application.

The main reasons for the low ratings the users stated were \textit{the workflow definition format being confusing} and the general unfamiliarity with debugging tools. 
While it would be easy to brush these points off as too subjective, the goal of the thesis was to create a tool that would be easy to use and intuitive for the user -
which makes all the user complaints very relevant.