\defcitealias{StateOfFtd20}{StateOfFtd20}
\defcitealias{StateOfJS21}{StateOfJS21}
\defcitealias{AnReVue22}{AnReVue22}
\section{Editor} \label{sec:editor}

As mentioned before, the \textit{Editor} should be a piece of software facilitating the process of creating and editing a web automation file.

While we designed the workflow definition format to be as readable and concise as possible, the strict JSON syntax makes the format hard to write manually.
Especially less technical users - such as the model user role \hyperref[UserUserRole]{\textit{User}} - could experience difficulties when trying to edit 
larger automation files. 
As stated in the \hyperref[requirements]{Requirements}, the \textit{Editor} should also provide performance-oriented features,
targetting more experienced users - such as the model \hyperref[DevUserRole]{\textit{Developer}}.

\subsection{Technologies}
In accordance with the nonfunctional requirements for the \textit{Editor}, the \textit{Editor} app should be user friendly and easy to use.
Due to its relatively lightweight nature, it is possible to implement the \textit{Editor} as a web application. 

This approach would be beneficial for several reasons - removing the need for installation, providing a cross-platform solution and speeding up the development and debugging process, to name a few.
Implementing the \textit{Editor} as a client-side web application in JavaScript seems like a sensible option also because it might later share a part of the codebase with the \textit{Runner} application.

\subsubsection{Frontend Framework}

Client-side web applications are now seldom developed using plain JavaScript - most developers utilize frontend frameworks and libraries for easier manipulation with the \ac{DOM} tree and state management \citepalias{StateOfFtd20}.

According to a popular 2021 JavaScript developer survey \citepalias{StateOfJS21}, the most popular \ac{JS} frontend frameworks among developers are \textit{React}, \textit{Angular} and \textit{Vue.js}.
While the popularity of the tools changes over time with new emerging technologies coming every year, because of their large following, the aforementioned tools have a large number of third-party libraries 
and modules. Compared to some less popular tools, these frameworks also have better documentation and are better debugged.

When comparing the frameworks against each other, \textit{React} comes off as the framework with the steepest learning curve while being only slightly less popular than \textit{Vue.js}, based on the GitHub stars of the project \citepalias{AnReVue22}.
While all the frameworks are utilizing the model of reusable components, \textit{Angular} and \textit{Vue.js} are taking the concept a little further with their internal \acs{HTML} templating systems.

\textit{React} is the only framework of those three utilizing \acs{JSX}, i.e. combining the \acs{HTML} syntax with the \acs{JS} syntax. 
While this might pose certain difficulties for developers learning this framework, it allows them to interleave the \acs{HTML} and \acs{JS} code in a way that is more readable and easier to maintain in the end.

Since the \textit{Editor}, given the \hyperref[requirements]{requirements}, should not require any advanced features of \textit{Angular} and \textit{Vue}, 
we can utilize the \textit{React} framework for the \textit{Editor}.

\subsubsection{}