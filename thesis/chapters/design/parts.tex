\section{Parts of the project}

As stated in the \hyperref[intro]{introduction} of this thesis, the goal of this thesis is to develop a clear, concise format for storing web automations
as well as tools for simplifying the work with the format. 

Given this assignment, it is only natural to first design the automation format, as the design of the tools for editing and debugging the automation files
largely depends on the format design itself. 
For the tools part, we can reuse the rather informal partition of the tools into \textit{Editor} and \textit{Runner}, following the idea from the \autoref{requirements} Requirements, 
as this still describes the two principal use cases of the toolkit.
The notional interface and the middle ground between those two parts (\textit{Editor} and \textit{Runner}) is then the workflow definition format, as both tools are designed to work with it, albeit in different ways.

The main parts of the project from now on are therefore the \textit{Format}, \textit{Editor} and \textit{Runner}.
The design of those three parts is discussed in the following sections separately.