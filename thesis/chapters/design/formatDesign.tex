\section{Workflow definition format}

As both the \textit{interpreter} and \textit{the editor} work directly with the files containing the workflow definitions, the first part of the project to be designed is the workflow definition format itself.

The workflow definition files should contain all the information needed to describe an arbitrary web-related workflow. 
The files in this format should also be parsable, human- and machine-readable and provide a simple yet powerful way of programming the web automations.

\subsection{Programming logic}

As the workflow definitions are computer programs of sorts, the first design decision needs to be what programming concepts will the file format implement.
To retain the steep learning curve and user-friendliness, this programming ``language'' also should not be too complicated.

\defcitealias{Sebesta2015}{Sebesta2015}

The trend in the current automation tools, such as \href{https://ifttt.com/}{IFTTT}, \href{https://zapier.com/}{Zapier} 
or \href{https://github.com/huginn/huginn}{Huginn} shows a rise in the popularity of \textit{declarative programming}.
Such languages and tools work with definitions of the desired results rather than describing the complete control flow. \citepalias{Sebesta2015}

Inspired by logic programming languages such as \textit{Prolog} - and its popular implementation \href{https://www.swi-prolog.org/}{SWI Prolog} - 
the workflow definition should contain a set of \textit{conditions} describing a possible state of the environment, connected to their respective \textit{reactions}, 
describing a sequence of actions to be carried out in case the condition applies.

% The similarity with \textit{Prolog} can be seen here, with the \textit{conditions} corresponding to \textit{Prolog's} fact \textit{heads} and \textit{reactions} to their respective \textit{bodies}.

\subsection{Conditions}
\label{conditions}

As stated before, the workflow definition format should allow the user to specify web environment-related conditions for running the automation steps.

Such conditions can be e.g. the browser visiting a certain \texttt{url}, the current page containing certain \texttt{selectors} or the current browser session having \texttt{cookies} set to specific values.

Moreover, the format should allow the user to combine the base conditions using \textit{boolean operators} to create more comprehensible and compact syntax.

Following through with the \textit{Prolog} comparison, the workflow definition could look something like this:

\begin{minipage}{0.95\linewidth}
\begin{lstlisting}[language=prolog, columns=spaceflexible]
    % X is denoting the current state of the browser
    % Y is to be unified with the next state

    nextState(X, Y) :- url(X, "https://jindrich.bar"),
                        % action to be 
                        % executed on 
                        % https://jindrich.bar

    nextState(X, Y) :- selector(X, "button"),
                        % action to be 
                        % executed if the current 
                        % page contains a button

    nextState(X, Y) :- cookies(X, "key", "value"),
                        % action to be 
                        % executed if the current 
                        % browser session has the
                        % `key` cookie for the  
                        % page set to `value`

    nextState(X, Y) :- url(X, "https://example.org"),
                        selector(X, "input"),
                        % action to be 
                        % executed in case of both
                        % conditions matching 
                        % (boolean AND example)

\end{lstlisting}
\end{minipage}

\emptyline

The conditions might also provide support for advanced functions such as wildcards or regular expressions.
Those would be particularly useful e.g. for URLs for targetting a specific domain, TLDs etc.

\subsection{Reactions}

The workflow definition format should also allow the user to specify the actions to be carried out when the respective condition matches.

Those can be e.g. \texttt{click}, \texttt{goto}, \texttt{scrapeData} and similar. 
The actions should be chainable, allowing the user to specify a set of actions to be executed sequentially, without additional condition matching between those.

Completing the \textit{Prolog-inspired} example from the previous section, the complete workflow definition would look like this:

\begin{minipage}{0.95\linewidth}
\begin{lstlisting}[language=prolog, columns=spaceflexible]
    % X is denoting the current state of the browser
    % Y is to be unified with the next state

    nextState(X, Y) :- url(X, "https://jindrich.bar"),
                        goto(X, Y, "https://example.org"), !.

    nextState(X, Y) :- selector(X, "button"),
                        click(X, Y, "button"), !.

    nextState(X, Y) :- cookies(X, "key", "value"),
                        click(X, Y, "logout"), !.

    nextState(X, Y) :- url(X, "https://example.org"),
                        selector(X, "input"),
                        fill(X, "input", "hello"), !.

\end{lstlisting}
\end{minipage}

\emptyline

The mock implementation of the workflow definition file in \textit{SWI-Prolog} is available as a \href{https://swish.swi-prolog.org/p/dwaim.pl}{snippet}\footnote{Available at \url{https://swish.swi-prolog.org/p/dwaim.pl}} in the \textit{Prolog} online execution environment \textit{Swish}. 

Please note that in this case, the \textit{Prolog} interpreter is actually taking the role of the workflow interpreter.

\multilinebox{
    \smallskip
    \textbf{Note:} This example also shows that the new state of the browser depends only on the preceding one. 
    
    Such quality, also called \textit{memorylessness}, or \textit{Markov property}, simplifies both the interpreter design and the programming concept itself.
    It might also allow for some optimizations utilizing parallel execution. 
    \smallskip
}

\subsection{Serialization}

\defcitealias{GTrends22}{GTrends22}
\defcitealias{Medium21}{Medium21}

Finally, the workflow definition needs to be physically stored in a file. 
As it would be rather counterproductive to develop a custom file format for storing the conditions and reactions, the workflow definitions might be stored using a host meta-format.

Based on the hierarchical nature of both \textit{condition-action} pairs and possibly recursive nature of the \textit{conditions} themselves, it would be only logical to store the definitions 
using a hierarchical data format like \textit{JSON}, \textit{XML}\footnote{\href{https://www.w3.org/TR/2006/REC-xml11-20060816/}{https://www.w3.org/TR/2006/REC-xml11-20060816/}} or \textit{YAML}\footnote{\href{https://yaml.org/}{https://yaml.org/}}.

Comparing these formats, \textit{JSON} comes out as the most popular \citepalias{GTrends22} and most space-saving \citepalias{Medium21}. 
While the advanced features of \textit{XML} are invaluable when working with complex structured data, it is perhaps too complicated for storing well-defined workflow definitions.

With YAML taking first place, JSON is also a runner-up in human readability.
While improving the file legibility, the indentation oriented nature of YAML makes it very prone to input errors - this problem is absent in JSON because of its bracket-oriented grammar.

For the reasons mentioned, the workflow definition format will be built upon JSON - a host format providing a simple, human-readable serialization for a structured schema of the definitions.

The JSON serialization of the workflow definition file might then look as follows:

\emptyline

\begin{minipage}{0.95\linewidth}
\begin{lstlisting}[language=json, columns=spaceflexible]
[
    {
        "conditions": {
            "url": "https://example.org"
        },
        "actions": [
            {
                "action": "goto",
                "args": ["https://jindrich.bar"]
            }
        ]
    },
    {
        "conditions": {
            "selector": "input"
        },
        "actions": [
            {
                "action": "fill",
                "args": ["input", "hello"]
            }
        ]
    }
]
\end{lstlisting}
\end{minipage}

\emptyline

While being more verbose, the JSON serialization is arguably more readable to a layman than the \textit{Prolog}-syntax pseudo implementation.

The root of the workflow definition is a \textit{JSON} array containing multiple objects, specifying the rules. 
Those have two keys, \texttt{conditions} and \texttt{actions}, defining the required conditions for the web browser environment and actions to be carried out in case the conditions apply, respectively.

The \texttt{conditions} object describes a valid state of the web browser using a set of predefined keys (\texttt{url}, \texttt{selector}, \texttt{cookies} and other).

The \texttt{action} object is an array of actions to be carried out. 
Every action is described using its name (\texttt{click}, \texttt{goto}, \texttt{fill}...) and an additional, action-specific set of arguments.

The action arguments are stored in an array. 
This also applies to singleton arguments, mainly to maintain consistency and improve the machine readability of the format.

\defcitealias{Mongo20}{Mongo20}

\subsubsection{Boolean operators}
As mentioned in subsection \ref{conditions} Conditions, the format should offer a way of combining the defined conditions using basic boolean operators.
Following other popular format based on JSON (or rather JavaScript objects), the workflow definition format might take inspiration from MongoDB query syntax \citepalias{Mongo20}.

In this query language, boolean operations are expressed using an object with a specific key, containing an array of operand objects.

\begin{center}
\begin{minipage}{0.95\linewidth}
    \begin{verbatim}

{$and: [{expression},{expression},...]}
{$or:  [{expression},{expression},...]}
{$not: {expression}}
    \end{verbatim}
\end{minipage}
\end{center}

This approach has a direct mapping to JSON syntax, which makes it very suitable for our use case.

\begin{minipage}{0.95\linewidth}
\begin{lstlisting}[language=json, columns=spaceflexible]
...
{
    "conditions": {
        "$and": [
            { "url": "https://example.org" },
            { "selector": "input" },
            { "selector": ".green" }
        ]
    },
    ...
},
...

\end{lstlisting}
\end{minipage}

Because of the associativity of those operations, the representation of the boolean operators AND and OR can have arbitrary arity.
Writing an empty \texttt{\$and} or \texttt{\$or} clause corresponds to an empty rule. 
Specifying only one subrule (using \texttt{\$and}/\texttt{\$or} as unary operators) defies the purpose of using those operators. 
With this being said, the format does not specify a required operator arity. 

\subsubsection{Regular expressions}

Following the same logic as with the boolean operators, the support for regular expressions requested in the subsection \ref{conditions} Conditions can take inspiration from MongoDB query operators \citepalias{Mongo20}.
In the MongoDB syntax, regular expressions are stored as objects with predefined keys - \texttt{"\$regex"} and \texttt{"\$options"}:

\begin{minipage}{0.95\linewidth}
    \begin{verbatim}
    
{ <field>: { $regex: /pattern/, $options: '<options>' } }
{ <field>: { $regex: 'pattern', $options: '<options>' } }
{ <field>: { $regex: /pattern/<options> } }
    \end{verbatim}
\end{minipage}

Regarding the types used, only the second mentioned alternative is directly translatable to JSON, as JSON cannot contain literal regular expressions. 
Therefore, storing the expressions as strings is the most fluent option.

\begin{minipage}{0.95\linewidth}
\begin{lstlisting}[language=json, columns=spaceflexible]
        
...
    { 
        "conditions": {
            "url": { "$regex": "http://.*" }
        } 
    },
...

\end{lstlisting}
\end{minipage}
    