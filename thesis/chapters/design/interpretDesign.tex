\section{Runner}\label{runnerDesign}

With the workflow definition format designed, we can now design the \textit{Runner} of this format.
As mentioned in the \autoref{requirements} Requirements, the \textit{Runner} should be a piece of software able to read, validate and execute the workflows defined in the aforementioned format.

\subsection{Components}

To facilitate later design decisions and understanding of the \ac{SW}, we divide the \textit{Runner} project into several independent parts.
These parts are:

\emptyline
\textbf{Browser}:  
The web browser to be automated.
To simplify the usage and installation of the solution, the \textit{Runner} should be able to work with stock (i.e. unpatched) versions of browsers, allowing the users to use their standard web browsers.

As seen in the \autoref{competition} Existing Solutions, most commercial web browsers already provide programmable interfaces (via CDP, RDP\dots).
The \textit{Runner} - browser communication can be further simplified by using low-level third-party libraries, also mentioned in the \autoref{competition} Existing Solutions.

\emptyline
\textbf{Workflow validator}:
A piece of software able to statically validate a given workflow definition file.
While it might provide various ways of validating the workflow definitions, the minimum is syntax validation, i.e. reading a file written 
in the format described above and telling whether it follows the definition of the format.
Optionally, the syntax validator might also provide descriptive error messages to communicate the problem with the user.

Given the programmable nature of the format, static ``code'' analysis might also take place here. 
While some workflow definition files might be syntactically correct, it is possible that they might contain logical errors.
The validator could then spot unreachable branches, suggesting reordering of the rules in the definition or suggesting updating the conditions.

\emptyline
\textbf{Workflow interpreter}:  
A piece of software comparing the current browser state with the conditions from the workflow definition, selecting the correct rule to be applied.
Furthermore, the \textit{Workflow interpreter} should also send the correct actions to the browser and ensure their execution went well.
In accordance with the requirement \textit{1.2.2.1.7}, the interpreter should inform the user in case of any exceptions.

\subsection{Programming language, libraries}

As mentioned above, the communication with the internal browser can be facilitated using a third-party library. 
This approach - compared to communicating with the web browser directly - leads to quicker development iteration and less cluttered code base, 
ultimately leading to a better tested software. 

Looking at the \hyperref[competition]{competition analysis}, the low-level automation libraries could be useful for this use case.
Both \textit{Puppeteer} and \textit{Playwright} offer a lightweight programmable interface by simply wrapping and unifying 
the debugging functionality of the web browsers (\ac{CDP} and alternatives).

As mentioned before, \textit{Puppeteer's} provides official support only for Chromium-based browsers, while \textit{Playwright} provides support for Chromium-, Firefox- and Webkit-based browsers alike.
For these mentioned reasons, the \textit{Workflow interpreter} will be using \textit{Playwright} as its backend library.

While \textit{Playwright} has bindings for different languages (\textit{JavaScript}, \textit{Python}, \textit{.NET} and \textit{Java} 
as of \today), the primary development is made in TypeScript (superset of JavaScript).

Given these facts, it would be beneficial to develop the \textit{Runner} also in Typescript.
This makes sense both because of the library support and the closeness of the language to the web environment - Typescript can be 
statically transpilled into Javascript, a popular client-side web programming language.

The utilization of Typescript also ensures type safety and better IDE support compared to regular JavaScript code.
This simplifies the development of the tool as well as the third-party adoption of the tools.