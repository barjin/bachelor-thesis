\section{Editor}
As described in the \autoref{sec:editor} \textit{Editor}, the workflow editor is implemented as a React Application.
The following section describes decisions made during the implementation of the \textit{Editor} application,
certain problems and their solutions.

\subsection{React}
While it is possible to set up a \textit{React} application as a plain Node.js application, it is not recommended.
Handling the bundler configuration is cumbersome and requires a respectable amount of knowledge about the React toolchain.
The React's signature \acs{JSX} syntax also requires configuring a transpiler (e.g. \texttt{babel}\footnote{\url{https://babeljs.io/}}), which adds another level of complexity.

For those reasons, the \textit{Editor} \textit{React} application has been initialized with \texttt{create-react-app}\footnote{\url{https://create-react-app.dev/}}. 
This is a \ac{CLI} tool for simple initialization of \textit{React} applications.
It mainly provides the \textit{Webpack} and \textit{Babel} configuration files and bootstraps the project with a template website.

\subsubsection{Configuring Webpack polyfills}

For validating the uploaded worklflow files, the \textit{Editor} application imports the \textit{Preprocessor} class from the \textit{Interpret} module.
While both the \textit{Editor} and \textit{Interpret} module are written in \acl{TS}, there are slight differences between \textit{Node.js} and browser code.

Most of those problems stem from the module resolution, as both \textit{Node.js} and browsers use slightly different approaches to module loading.
While this is covered by Webpack and Babel in most cases, importing the \textit{Interpret} module initially caused errors.

This is because since the version \texttt{5.0.0}, \textit{Webpack} no longer provides polyfills for the core \textit{Node.js} modules, such as \texttt{path} used by the \textit{Interpret} module.
The \texttt{path} module is not used by the validation feature of the \textit{Preprocessor}, so the polyfill could be simply disabled.
However, doing so requires a manual update of the \textit{Webpack} configuration.

Since the \textit{Editor} has been created with \texttt{create-react-app}, the \textit{Webpack} configuration \texttt{webpack.conf.js} is contained in the \verb|node_modules/react-scripts| directory.
While it might be possible to modify the \texttt{webpack.conf.js} file directly, it is not recommended - the \verb|node_modules| directory is used to save the packages downloaded from the NPM registry.
Updating the \texttt{react-scripts} package would then reset all changes made to the configuration file.
Furthermore, because of it's dynamic nature, the \verb|node_modules| directory is omitted from the \texttt{git} version control.

While the official \texttt{create-react-app} guide suggests to perform \texttt{eject}, i.e. to export the configuration files for manual maintenance, this is a borderline unsafe step. 
The \texttt{eject} script performs irreversible changes to the package structure, and forces the user to maintain the configuration and dependencies themselves from then on.

To retain the simplicity of automatic project management, the project now uses \texttt{react-app-rewired}\footnote{\url{https://www.npmjs.com/package/react-app-rewired}}.
Published as an \texttt{npm} package, \texttt{react-app-rewired} is an drop-in replacement of \texttt{create-react-app}, which lets the developer to sideload custom configuration files, while still maintaining the base configuration.

\subsection{Improving the \acs{UX}}
Given the \hyperref[requirements]{nonfunctional requirements} for the \textit{Editor}, the \textit{Editor} application should adhere to the best \ac{UI}/\ac{UX} practices and provide a steep learning curve.
This is manifested multiple times in the \textit{Editor} application itself.

\subsubsection{Drag \& Drop}
Since the main control elements in the \textit{Editor} are modular blocks, allowing the user to use drag\&drop controls e.g. for reordering the blocks seems like the superior solution in terms of UX.

The drag\&drop feature is used for reordering the blocks in the \textit{Editor} application.
While it would be possible to drag the entire blocks around the application window, it is not recommended due to UX reasons.
This is solved in the \textit{Editor} application by collapsing all the blocks when the drag\&drop action is initiated.
This helps the user to focus on the actual meaning of the reordering action, and not on the content of the blocks.

The implementation of the drag\&drop feature is based on the \texttt{react-dnd}\footnote{\url{https://react-dnd.github.io/react-dnd/}} library.
While the authors of this library offer an official example of reordering a list of block elements, the \textit{Editor} implementation is not based on this example due to the mentioned features,
which are not provided by the example solution.

\subsubsection{Argument type suggestions}
The \textit{Editor} application generates a workflow interpretable by the \textit{Runner} application, which is internally using the \textit{Playwright} library.
The function signatures of the workflow reaction steps depend on the \textit{Playwright} library, as most of the supported steps are mirrored from the \texttt{Page} class methods.
While the user might set the types of the arguments themselves, the \textit{Editor} application aims to provide a simple way of creating the web automations.

For this reason, the \textit{Editor} application provides a prefilled list of arguments with the correct types and matching input elements,
which correspond to the \textit{Playwright} \texttt{Page} class methods and present the correct usage of the arguments to the user.
The range of optional arguments is hand-picked for every command to provide the best variability while still maintaining an exceptional level of user-friendliness.

\subsubsection{Real-time validation}
On top of the mentioned argument type suggestion feature, the \textit{Editor} application also validates most of the user inputs based on their context in the workflow definition.
Just like the previously mentioned features, the only purpose of this is enhancement of \acs{UX} and a user guide; a workflow definition with inputs marked as invalid still can be executed.