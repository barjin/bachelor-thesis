\defcitealias{DockerRun}{DockerRef}

\section{Administrator documentation}

The following section contains the administrator documentation of the project.
It contains installation instructions and a short troubleshooting guide for setting up a server for the \textit{Editor} and \textit{Runner} applications. 
It also contains a list of system requirements on the server where the applications are to be deployed

\subsection{Installation instructions}

The \textit{Editor} and \textit{Runner} application are both packaged in the \href{https://www.docker.com/}{Docker} image \texttt{barjin/wbr}.
This Docker image represents a complete server setup, including all the dependencies and the \textit{Editor} and \textit{Runner} applications.

The web server providing the user interface is running on port \texttt{4321} of the Docker container, which is also the only port utilized by the \textit{Editor} and \textit{Runner} applications.

Running the container with the \texttt{barjin/wbr} image then requires only one command:
\begin{verbatim}
    docker run -p 4321:4321 -d barjin/wbr
\end{verbatim}
where \texttt{4321} is the port number used by the \textit{Editor} and \textit{Runner} applications.

The \texttt{-d} option instructs Docker to run the container in so-called `detached mode', which means that the container is not terminated when the command finishes. \citepalias{DockerRun}
In case the user wants to stop the container, they can use the \texttt{docker stop} command with the container ID.

In case the user never ran the \texttt{docker run} nor the \texttt{docker pull} command before, the \textit{Docker daemon} first downloads the \texttt{barjin/wbr} image from the Docker Hub.
Note that this can cause a delay of several seconds and consume a certain amount of bandwidth (ca. TODO MiB).

The Docker image can also be built from the attached Dockerfile by invoking the \texttt{docker build} command in the root folder of the project repository.

\subsection{Build instructions}
Besides the \textit{Docker} image, source files of both applications are also available in the project's GitHub repository. 
The \texttt{package.json} files for the packages \texttt{wbr-interpret} and \texttt{wbr-editor} contain the build instructions for the respective applications,
as well as the required dependencies.

Both packages can be built by invoking the \texttt{npm run build} command in the root folder of the package folders.
After the build process, the tools can be started by calling \texttt{npm start} in the respective package folders.

Please note that building the packages individually is not the recommended way of running the application.
The author of this work holds no responsibility for the results of the build process.

Running the project in the \textit{Docker} container is the recommended way of running the application.