\defcitealias{DockerRun}{DockerRef}

\section{Administrator documentation} \label{adminDocs}

The following section contains the administrator documentation of the project.
It contains installation instructions and a short troubleshooting guide for setting up a server for the \textit{Editor} and \textit{Runner} applications. 
It also contains a list of system requirements on the server where the applications are to be deployed.

Aside from this, this section also contains installation instructions for the \textit{Runner} \texttt{npm} package, enabling the user to create software able to execute and validate the workflow definition files.

\subsection{Editor application}

% TODO: initial words

\subsubsection{Installation instructions}

The \textit{Editor} and \textit{Runner} application are both packaged in the \href{https://www.docker.com/}{Docker} image \texttt{barjin/wbr}.
This Docker image\footnote{Tag \texttt{barjin/wbr:final}, \texttt{sha256:b9e237a3ccf619f4a9b36e6584191bf}} represents a complete server setup, including all the dependencies and the \textit{Editor} and \textit{Runner} applications.
To run a Docker image, the workstation must have \textit{Docker} installed. 
Aside from meeting the \textit{Docker} system requirements, the \textit{Editor} Docker image presses no other requirements on the workstation.

The web server providing the user interface is running on port \texttt{8080} of the Docker container, which is also the only port utilized by the \textit{Editor} and \textit{Runner} applications.
On a system with \texttt{docker} installed, running the container requires only one command:

\begin{verbatim}
            docker run -p HOSTPORT:8080 -d barjin/wbr
\end{verbatim}
where \texttt{HOSTPORT} is the port number used by the \textit{Editor} and \textit{Runner} applications on the host machine.
After running this command, the user interface should be now available at \texttt{http://localhost:HOSTPORT/}.

The \texttt{-d} option instructs Docker to run the container in so-called `detached mode', which means that the container is not terminated when the command finishes. \citepalias{DockerRun}
In case the user wants to stop the container, they can use the \texttt{docker stop} command with the container ID.

In case the user never ran the \texttt{docker run} nor the \texttt{docker pull} command before, the \textit{Docker daemon} first downloads the \texttt{barjin/wbr} image from the Docker Hub.
Note that this can cause a delay of several seconds and consume a certain amount of bandwidth (ca. 200 MiB).

The Docker image can also be built from the attached Dockerfile by invoking the \texttt{docker build} command in the root folder of the project repository.

Please note that the \textit{Editor} application allows the users to run arbitrary code on the server.
This is a security risk, and the user is advised to only share their instance of \textit{Editor} server with trusted users.

\subsubsection{Build instructions}
Besides the \textit{Docker} image, all the software source files are also available in the project's GitHub repository.\footnote{Tag \texttt{v1.0}, commit hash \texttt{bf45528225e3b9fc05963d75}}

The \texttt{package.json} files for the packages \texttt{wbr-interpret}, \texttt{wbr-editor} and \texttt{wbr-cloud} contain the build instructions for the respective applications,
as well as the required dependencies.

\multilinebox{
    \textbf{Note:} \texttt{npm} is required for building the \textit{Editor} and \textit{Runner} applications.
    Before building the applications, run
    \begin{center}
    \texttt{npm run confBuildDeps}
    \end{center}
    command in the root folder of the project repository.
    This installs the correct versions of build dependencies in the correct order. 
    Running the \texttt{build} command without the dependency installation - or installing the build dependecies in a different way - can result in build errors and/or unexpected behavior.
}

The build process is managed by the \textit{Turborepo}\footnote{\url{https://turborepo.org/}} build system. 
\textit{Turborepo} allows for faster build times and less wordy build configurations by utilizing a \texttt{make}-like approach 
to the build process. It caches the built files and allows for incremental builds and faster rebuilds. 
It also constructs a dependency graph of the source packages by reading their \texttt{package.json} files, which is used to determine the order in which the packages are built.

The \texttt{turbo} build is invoked by the \texttt{npm run build} command in the root folder of the repository.
Invoking the \texttt{npm start} command in the root folder of the repository after the build starts the \textit{Editor} application server.