\section{User documentation}

The following section contains user documentation for both the \textit{Editor} application and the \textit{Runner} module with described actions and user scenarios, following the scenarios from the \autoref{analysis} Analysis.



\subsection{Runner package}
While the \textit{Editor application} allows the user to execute automations and provides simple debugging features, this might not be enough.
To use the features of the \textit{Runner} programatically, it is possible to install the \textit{Runner} module as an \texttt{npm} package.

This is simply done by running the command
\begin{center}
\verb|npm i -s @wbr-project/wbr-interpret|
\end{center}
in one's \textit{Node} project folder. 
This installs the module into the current project's dependency tree.

The module package contains extensive \acs{TS} typings, further simplifying the development.
The type definition in this package also contains typings for the workflow definition files, allowing the user to update the workflow files manually, utilizing the \acl{TS} \acs{IDE} suggestions.

Using the module from one's code is then as simple as importing the module with
\begin{verbatim}
    import Interpreter from 'wbr-interpret';
    - or -
    const Interpreter = require('wbr-interpret');
\end{verbatim}

The \verb|Interpreter| imported from the module is a constructor for the \textit{Interpreter} class.
This class, together with the \verb|Preprocessor| class provides all the features of the \texttt{wbr-interpret} module.

\subsubsection{Interpreter}

As stated above, the default export of the \texttt{wbr-interpret} package is the \textit{Interpreter} class.
This class implements the main part of the automation execution.

The intended usage of the interpreter class is
% TODO 

Example workflow files demonstrating different features of the \texttt{wbr-interpret} package are available in the \textit{GitHub} repository.

\subsubsection{Preprocessor}

The \texttt{Preprocessor} class contains static methods for validation, analysis and initialization of the workflow file definitions.
This class is exported from the \texttt{wbr-interpret} package as a named export \texttt{Preprocessor}.

The public methods provided by this class are:
\smallskip

\texttt{Preprocessor.validateWorkflow(workflow)} - a static method for checking the validity of the provided workflow. 
Returns a \texttt{string} description of the workflow definition syntax violation.
In case of a valid workflow, this method returns \texttt{undefined}.
\smallskip

\texttt{Preprocessor.getParams(workflow)} - a static method for extracting the parameter names from the provided workflow definition.
Returns an array of workflow parameter names, based on the parameter structures in the provided workflow definition.
These can be used for querying the user about the desired values of those parameters - the values must be provided before the workflow execution.
\smallskip

\texttt{Preprocessor.extractSelectors(workflow)} - a static method for gathering all the selector names from the workflow. 
Returns an array of the selectors used in the workflow's conditionals.
Is used to improve the performance of the workflow interpreter, as described in the \autoref{browsercom} Browser communication.
\smallskip

\texttt{Preprocessor.initWorkflow(workflow, parameters)} - a static method for initializing the workflow definition file. 
Accepts an uninitialized workflow definition and -optionally- an object mapping parameter names to the desired values.
Returns a copy of the provided workflow definition initialized with the provided parametets to be used with the \textit{Intepreter} class.
\smallskip

Example usage of all the mentioned methods is here:

\begin{minipage}{0.95\linewidth}
    \begin{lstlisting}[language=javascript, columns=spaceflexible]
import { Preprocessor as P } from 'wbr-interpret';

const workflow = { x: { $param: "valueOfX" } };
const e = P.validate(workflow);
// e -> '"workflow" is required'

const params = P.getParams(workflow);
// params -> ["valueOfX"]

const sel = P.extractSelectors(workflow);
// sel -> []

const wfInit = P.initWorkflow(workflow, {"valueofX": "abc"});
// wfInit -> { x: abc }
    \end{lstlisting}
\end{minipage}