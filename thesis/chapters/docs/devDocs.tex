\clearpage
\section{Developer documentation} \label{devDocs}

This section contains the developer documentation for the \texttt{wbr-interpret} module and the \textit{Editor} application.
The documentation presented here describes the main design principles of both pieces of software and how they are implemented.

\subsection{wbr-interpret}

To run the web automations made in the \textit{Editor} application programatically, 
it is possible to install the \texttt{wbr-interpret} module as an \texttt{npm} package.

This is simply done by running the command
\begin{center}
\verb|npm i -s @wbr-project/wbr-interpret|
\end{center}
in one's \textit{Node} project folder. 
This installs the module into the current project's dependency tree.

The module package contains extensive \acs{TS} typings, further simplifying the development.
The type definition in this package also contains typings for the workflow definition files, allowing the user to update the workflow files manually, 
utilizing the \acl{TS} \acs{IDE} suggestions and validation.

\subsubsection{Module usage}

Using the \texttt{wbr-interpret} module from one's code is quite simple.
The main \textit{Interpreter} class provides only two public methods - \texttt{run()} and \texttt{stop()}.
The intended usage is then demonstrated in the following example.\footnote{This example uses CommonJS module syntax.}

\begin{lstlisting}[language=javascript]
const Interpreter = require('wbr-interpret');
const { chromium } = require('playwright');

const workflow = {...}

(async () => {
    const browser = await chromium.launch();
    const page = await browser.newPage();
    const interpreter = new Interpreter(workflow);
    
    await interpreter.run(page, { paramName: 'paramValue' });
    await page.close();
    await browser.close();
})();
\end{lstlisting}

As shown in the example, the \textit{Interpreter} class accepts the workflow definition in the constructor. The \texttt{run()} method of the \textit{Interpreter} takes a Playwright \texttt{page} object as an argument and a \texttt{parameters} object as an optional second argument.

\subsubsection{The package structure}

The majority of the logic in the \texttt{wbr-interpret} module is implemented in two main files: \texttt{interpret.ts} and \texttt{preprocessor.ts}.
Both of those files also import utility functions from the files located in the \texttt{utils/} directory.

\texttt{interpret.ts} contains the main logic for the workflow execution. 
While the \texttt{wbr-interpret} module mostly imitates the behaviour of the \textit{Playwright}'s \texttt{Page} class during the workflow execution,
it also provides custom functions or overrides for the existing \texttt{Page} class functions, updating their behaviour to suit the interface of \texttt{wbr-interpret} better.
The \texttt{interpret.ts} file contains implementation for those methods as well. 

\texttt{preprocessor.ts} contains the validation logic for the workflow definition files, as well as methods for workflow initialization.
The methods from this file can be used for runtime validation of the workflow definition files
and other preprocessing analysis tasks.

More detailed descriptions of the code in the \texttt{wbr-interpret} module can be found in the Attachment \ref{atta:interpretCode}.

% While the \textit{Editor application} allows the user to execute automations and provides simple debugging features, this might not be enough.
% To use the features of the \textit{Runner} programatically, it is possible to install the \textit{Runner} module as an \texttt{npm} package.

% Using the module from one's code is then as simple as importing the module with
% \begin{verbatim}
%     import Interpreter from 'wbr-interpret';
%     - or -
%     const Interpreter = require('wbr-interpret');
% \end{verbatim}

% The \verb|Interpreter| imported from the module is a constructor for the \textit{Interpreter} class.
% This class, together with the \verb|Preprocessor| class provides all the features of the \texttt{wbr-interpret} module.

% \subsubsection{Interpreter}

% As stated above, the default export of the \texttt{wbr-interpret} package is the \textit{Interpreter} class.
% This class implements the main part of the automation execution.

% The intended usage of the interpreter class is
% % TODO 

% Example workflow files demonstrating different features of the \texttt{wbr-interpret} package are available in the \textit{GitHub} repository.

% \subsubsection{Preprocessor}

% The \texttt{Preprocessor} class contains static methods for validation, analysis and initialization of the workflow file definitions.
% This class is exported from the \texttt{wbr-interpret} package as a named export \texttt{Preprocessor}.

% The public methods provided by this class are:
% \smallskip

% \texttt{Preprocessor.validateWorkflow(workflow)} - a static method for checking the validity of the provided workflow. 
% Returns a \texttt{string} description of the workflow definition syntax violation.
% In case of a valid workflow, this method returns \texttt{undefined}.
% \smallskip

% \texttt{Preprocessor.getParams(workflow)} - a static method for extracting the parameter names from the provided workflow definition.
% Returns an array of workflow parameter names, based on the parameter structures in the provided workflow definition.
% These can be used for querying the user about the desired values of those parameters - the values must be provided before the workflow execution.
% \smallskip

% \texttt{Preprocessor.extractSelectors(workflow)} - a static method for gathering all the selector names from the workflow. 
% Returns an array of the selectors used in the workflow's conditionals.
% Is used to improve the performance of the workflow interpreter, as described in the \autoref{browsercom} Browser communication.
% \smallskip

% \texttt{Preprocessor.initWorkflow(workflow, parameters)} - a static method for initializing the workflow definition file. 
% Accepts an uninitialized workflow definition and -optionally- an object mapping parameter names to the desired values.
% Returns a copy of the provided workflow definition initialized with the provided parametets to be used with the \textit{Intepreter} class.
% \smallskip

% Example usage of all the mentioned methods is here:

% \begin{minipage}{0.95\linewidth}
%     \begin{lstlisting}[language=javascript, columns=spaceflexible]
% import { Preprocessor as P } from 'wbr-interpret';

% const workflow = { x: { $param: "valueOfX" } };
% const e = P.validate(workflow);
% // e -> '"workflow" is required'

% const params = P.getParams(workflow);
% // params -> ["valueOfX"]

% const sel = P.extractSelectors(workflow);
% // sel -> []

% const wfInit = P.initWorkflow(workflow, {"valueofX": "abc"});
% // wfInit -> { x: abc }
%     \end{lstlisting}
% \end{minipage}