\clearpage
\section{Developer documentation} \label{devDocs}

This section contains the developer documentation for the \texttt{wbr-interpret} module and the \textit{Editor} application.
The documentation presented here describes the main design principles of both pieces of software and how they are implemented.

\subsection{wbr-interpret}

To run the web automations made in the \textit{Editor} application programatically, 
it is possible to install the \texttt{wbr-interpret} module as an \texttt{npm} package.

This is simply done by running the command
\begin{center}
\verb|npm i -s @wbr-project/wbr-interpret|
\end{center}
in one's \textit{Node} project folder. 
This installs the module into the current project's dependency tree.

The module package contains extensive \acs{TS} typings, further simplifying the development.
The type definition in this package also contains typings for the workflow definition files, allowing the user to update the workflow files manually, 
utilizing the \acl{TS} \acs{IDE} suggestions and validation.

\subsubsection{Module usage}

Using the \texttt{wbr-interpret} module from one's code is quite simple.
The main \textit{Interpreter} class provides only two public methods - \texttt{run()} and \texttt{stop()}.
The intended usage is then demonstrated in the following example.\footnote{This example uses CommonJS module syntax.}

\begin{lstlisting}[language=javascript]
const Interpreter = require('wbr-interpret');
const { chromium } = require('playwright');

const workflow = {...}

(async () => {
    const browser = await chromium.launch();
    const page = await browser.newPage();
    const interpreter = new Interpreter(workflow);
    
    await interpreter.run(page, { paramName: 'paramValue' });
    await page.close();
    await browser.close();
})();
\end{lstlisting}

As shown in the example, the \textit{Interpreter} class accepts the workflow definition in the constructor. The \texttt{run()} method of the \textit{Interpreter} takes a Playwright \texttt{page} object as an argument and a \texttt{parameters} object as an optional second argument.

\subsubsection{The package structure}

The majority of the logic in the \texttt{wbr-interpret} module is implemented in two main files: \texttt{interpret.ts} and \texttt{preprocessor.ts}.
Both of those files also import utility functions from the files located in the \texttt{utils/} directory.

\texttt{interpret.ts} contains the main logic for the workflow execution. 
While the \texttt{wbr-interpret} module mostly imitates the behaviour of the \textit{Playwright}'s \texttt{Page} class during the workflow execution,
it also provides custom functions or overrides for the existing \texttt{Page} class functions, updating their behaviour to suit the interface of \texttt{wbr-interpret} better.
The \texttt{interpret.ts} file contains implementation for those methods as well. 

\texttt{preprocessor.ts} contains the validation logic for the workflow definition files, as well as methods for workflow initialization.
The methods from this file can be used for runtime validation of the workflow definition files
and other preprocessing analysis tasks.

More detailed descriptions of the code in the \texttt{wbr-interpret} module can be found in the Attachment \ref{atta:interpretCode}.

\subsection{wbr-editor}

The React application \texttt{wbr-editor} is a web application that allows the user to create and edit web automation workflows.
The following section describes the main design principles of the \texttt{wbr-editor} application.

To run the development server, the user requires to run the following sequence of commands, while in the \textit{wbr-editor} project folder:
\begin{verbatim}
$ npm i -s
$ npm start
\end{verbatim}

This runs the \textit{react-scripts} development server on a free port (typically \texttt{3000}).
The user documentation for the web application is available in the \autoref{userDocs} User documentation.

Please note that the \texttt{react-scripts} development server does not contain the backend logic for the automation execution
and is meant to be run only during development of the web application.

\subsubsection{Package structure}

The \texttt{wbr-editor} package is structured as a regular \textit{React} application. 
All the source code files are located in the \texttt{src/} directory.
The static files (i.e. images) are located in the \texttt{public/} directory.

Closer description of the file contents can be found in the Attachment \ref{atta:editorCode}.

\subsection{wbr-cloud}

