\section{Requirements}
\label{requirements}

The following section describes functional and non-functional requirements for the toolkit project, 
based on the requirements of the user roles described in the \autoref{userroles} User roles.

For clarity, let us divide the toolkit project into individual tools serving different purposes.
As the main purposes of the toolkit are \textit{creating}, \textit{editing} and \textit{running web automations},
we can talk about the \textit{Editor} and the \textit{Runner} parts separately.

\subsection{Editor}

The \textit{Editor} is the part of the toolkit allowing the users to create and edit the web automations.
It should provide a user-friendly way of doing so while not restricting the more advanced users.

\smallskip

\subsubsection{Functional Requirements}

\begin{enumerate}[label=\thesubsection.1.\arabic*]
    \item The \textit{Editor} must allow the user to create a valid workflow file.
    \item The \textit{Editor} must enable the user to upload a valid workflow file into the \textit{Editor}, 
    making it possible to edit this file. If the uploaded file is not a valid workflow file, the \textit{Editor}
    must reject it.
    \item The \textit{Editor} must allow the user to edit the workflow file. 
    No user-induced change to the file shall corrupt the valid file syntax.
    \item The \textit{Editor} must allow the user to export a valid workflow file. 
    This exported file must be readable by the \textit{Runner}.
\end{enumerate}

\subsubsection{Nonfunctional requirements}

\begin{enumerate}[label=\thesubsection.2.\arabic*]
    \item The user interface of the \textit{Editor} shall be user-friendly and adhere to the best \ac{UI} practices.
    \item The \textit{Editor} shall contain hints for inexperienced users. 
    The \textit{Editor} shall allow more experienced users to disable those hints.
\end{enumerate}

\subsection{Runner}

The \textit{Runner} is the part of the toolkit providing support for executing the automations made with the \textit{Editor}.
It should provide a safe and optimized way for running the automations as well as a comprehensive user interface.

\subsubsection{Functional Requirements}

\begin{enumerate}[label=\thesubsection.1.\arabic*]
    \item The \textit{Runner} must allow the user to execute given valid automations. 
    If the provided automation is not valid, the \textit{Runner} must refuse such automation, notifying the user.
    \item If the provided automation provided to the \textit{Runner} is not valid, 
    the \textit{Runner} must provide the user with detailed information about the errors.
    \item The \textit{Runner} must allow the user to observe the automation run.
    \item The \textit{Runner} must enable the user to interrupt the automation execution in an arbitrary moment.
    \item The \textit{Runner} must provide options for debugging the automations. 
    These must expose additional information, allowing the more experienced users to develop the automations faster.
    \item The \textit{Runner} must enable the user to start automation with an arbitrary context (webpage, webpage state).
    \item The \textit{Runner} must inform the user of any runtime errors. 
    \item The \textit{Runner} must expose a programmable interface to allow for a simple third-party adoption.
    Furthermore, the \textit{Runner} must also log all errors appropriately.
\end{enumerate}

\subsubsection{Nonfunctional requirements}

\begin{enumerate}[label=\thesubsection.2.\arabic*]
    \item The \textit{Runner} shall implement the automation execution in an optimized way.
    \item The \textit{Runner}'s automation validator shall provide a detailed description
\end{enumerate}