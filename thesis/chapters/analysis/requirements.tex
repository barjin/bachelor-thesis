\section{Functional Requirements}

The following section describes functional requirements for \textit{the editor} and \textit{the interpreter}. 

\subsection{Editor}

The editor \textbf{must}:
\begin{itemize}
    \item Allow the user to create a valid workflow file.
    \item Allow the user to add, modify, remove and rearrange the rules within the workflow file.
    \item Let the user add, modify and remove logical conditions in the rule definition.
    \item Allow the user to work with nested logical conditions using boolean logic.
    \item Let the user specify a sequence of actions to be carried out when the respective rule is matched.
    \item Let the user export the created workflow as a valid JSON file processable by \textit{the interpreter}.
    \item Provide a user-friendly interface for all the mentioned actions.
    \item Contain a graphical tutorial, leading the user through the first steps of workflow creation.
    \item Allow the user to upload the workflow file into the editor, making it possible to edit it.
    \item Provide simple syntax checking for the uploaded files, rejecting invalid files.
\end{itemize}

\subsection{Interpreter}

The interpreter is a computer program, capable of reading and executing the workflow definition files.
The interpreter \textbf{must}:
\begin{itemize}
    \item Be able to syntax-validate a user-supplied workflow file.
    \item Provide a comprehensive API.
    \item Be published as an installable package, allowing simple adoption by third-party \acs{SW} developers.
    \item Be able to execute a user-supplied workflow definition file.
    \item Output well-formated workflow results in a machine-readable format.
\end{itemize}