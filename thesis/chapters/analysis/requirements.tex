\defcitealias{UIDesign}{UIDesign}

\section{Requirements}
\label{requirements}

The following section describes functional and non-functional requirements for the toolkit project, 
based on the requirements of the user roles described in the \autoref{userroles} User roles.

For clarity, let us divide the toolkit project into individual tools serving different purposes.
As the main purposes of the toolkit are \textit{creating}, \textit{editing} and \textit{running web automations},
we can talk about the \textit{Editor} and the \textit{Runner} parts separately.

\emptyline
\subsection{Editor}

The \textit{Editor} is the part of the toolkit allowing the users to create and edit the web automations.
It should provide a user-friendly way of doing so while not restricting the more advanced users.

The goal of the \textit{Editor} is not to exhaustively support all the features of the workflow definition syntax, but to provide a simple and intuitive way of creating and editing web automations.

\smallskip

\subsubsection{Functional Requirements}

\begin{enumerate}[label=\thesubsection.1.\arabic*]
    \item The \textit{Editor} must allow the user to create a valid workflow file.
    \item The \textit{Editor} must enable the user to upload a valid workflow file into the \textit{Editor}.
    \item If the uploaded file is not a valid workflow file, the \textit{Editor} must reject it.
    \item The \textit{Editor} must allow the user to edit the workflow file. 
    No user-induced change to the file shall corrupt the valid file syntax.
    \item The \textit{Editor} must allow the user to export a valid workflow file. 
    This exported file must be readable by the \textit{Runner}.
    \item The \textit{Editor} must interface the \textit{Runner}, allowing the user to test run the automations.
    \item During the test run, the \textit{Editor} must display the automation results in a human-readable way.
\end{enumerate}

\subsubsection{Nonfunctional requirements}

\begin{enumerate}[label=\thesubsection.2.\arabic*]
    \item The user interface of the \textit{Editor} shall adhere to the best \ac{UI} practices.\citepalias{UIDesign}
    \item The \textit{Editor} shall contain example workflow files for the user to study and to showcase the capabilities of the toolkit.
\end{enumerate}

\clearpage
\subsection{Runner}

The \textit{Runner} is the part of the toolkit providing support for executing the automations made with the \textit{Editor}.
It should provide a safe and optimized way for running the automations as well as a comprehensive user interface.

\subsubsection{Functional Requirements}

\begin{enumerate}[label=\thesubsection.1.\arabic*]
    \item The \textit{Runner} must allow the user to execute given valid automations. 
    \item If the provided automation is not valid, the \textit{Runner} must refuse such automation, notifying the user.
    \item If the automation provided to the \textit{Runner} is not valid, 
    the \textit{Runner} must provide the user with detailed information about the errors.
    \item The \textit{Runner} must allow the user to observe the automation run.
    \item The \textit{Runner} must provide the user with additional information about the automation run. 
    \item The \textit{Runner} must enable the user to interrupt the automation execution at an arbitrary moment.
    \item The \textit{Runner} must inform the user of any runtime errors. 
    Furthermore, the \textit{Runner} must also log all errors appropriately.
    \item The \textit{Runner} must expose a programmable interface to allow for a simple third-party adoption.
\end{enumerate}

\subsubsection{Nonfunctional requirements}

\begin{enumerate}[label=\thesubsection.2.\arabic*]
    \item The \textit{Runner} shall implement the automation execution in an optimized way.
    \item The installation of the \textit{Runner} shall be simple, allowing for quick adoption of the \acl{SW}.
\end{enumerate}