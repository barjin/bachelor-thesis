\defcitealias{PPatch21}{PPatch21}
\defcitealias{PReadme22}{PReadme22}

\section{Existing solutions}
As of now, there are already numerous solutions for automating web actions on the market. 
A majority of those uses existing web browsers and offer a programmable interface for simulating user input.

\textbf{Cypress} is a \ac{E2E} Javascript testing framework containing various assertions for \ac{QA} testing of webpages.
It supports multiple web browsers and offers its own UI and toolkit for test programming and running. 
Due to its strong orientation towards testing, it does not provide much methods for data extraction and crawling.

\textbf{Selenium WebDriver} is a fairly popular tool among web UI testers, as it offers a wide variety of methods for \ac{QA} testing.
Distributed as a multilanguage library, Selenium implements a high-level interface for controlling web browsers from code.
% Aside from regular commercial web browsers, Selenium also implements interfaces for PhantomJS and HTMLUnit, both headless scriptable browsers used as a lightweight alternative to regular browsers.

\textbf{Puppeteer} is a low-level library used for web browser automation. 
Unlike Cypress and Selenium, Puppeteer supports Chrome (or Chromium) as its only backend browser as of now (\today).

The communication with the browser is implemented via WebSockets and the DevTools Protocol, a Chromium-specific set of commands.
This allows Puppeteer to exceed Selenium both in stability and performance, sporting up to $17\%$ speedup in benchmarks \citepalias{Chck21}.

\textbf{Playwright} is another low-level library multilanguage library offering programmable ways of controlling web browser.
For browser communication, Playwright uses similar technology as Puppeteer, unlike Puppeteer, Playwright supports multiple commercial browsers (Chromium, Firefox, Webkit as of \today).

Due to differences between browsers and partial incompatibility of protocols, Playwright is distributed with patched versions of Firefox and Webkit \citepalias{PPatch21}.
Stock versions of Chromium-based browsers (Google Chrome, Microsoft Edge) are supported. \citepalias{PReadme22}

\emptyline

All the aforementioned examples however require programming, which can mean a significant barrier to entry for beginners
Beside these examples, there are also other solutions, allowing the users to create, manage and execute automated workflows using higher-level UI actions.

\emptyline

\textbf{Dexi.io} is one of such services. 
Accessible as a web application, it provides the user with a web-based browser recorder, removing both the need for programming and package installation.
Dexi.io \ac{GUI} editor allows for creating branches based on user-specified conditions.

Dexi.io also serves as an execution platform for these recordings. 
While the recordings are exportable in a JSON-based format, the definition of this format is closed, effectively causing a vendor lock-in.

The recorder app suffers from problems stemming from its web nature, namely \acs{CORS}-related issues, being targetted by anti-scraping measures and worse responsiveness.

\textbf{Browse.ai} serves similar purpose. 
Utilizing a Chrome-only browser extension for workflow recording, Browser.ai offers arguably better \ac{UX} than Dexi.io with more accurate web page representation.

The execution of Browser.ai recordings is available only through the associated web service without any export option, causing even stronger vendor lock-in than Dexi.io.

\textbf{Chrome Recorder} is a preview feature of the Google Chrome web browser (as of \today).
Embedded into the browser itself, the Chrome recorder provides the best performance and responsiveness of the mentioned examples. 

Since this browser feature is targetted mainly at the \ac{QA} testing community, the recorder offers detailed perfomance measurement features.
For the same reason, data extraction methods are completely missing, rendering the Chrome Recorder unusable for web scraping use cases.

The created recording is exportable as a JavaScript code utilizing the Puppeteer library.