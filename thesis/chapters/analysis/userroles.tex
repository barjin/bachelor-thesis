\section{User roles} \label{userroles}

In the first section of the analysis, we describe the typical users of such a toolkit.
Users may have different requirements based on their level of expertise and knowledge.

While the toolkit should be accessible and user-friendly enough to allow beginners to
create web automations with ease; it also should provide the more experienced users with 
an advanced functionality required for handling specific use cases. 

For clarity, we describe only two user roles with a significant difference in skill and knowledge.
Please note that these roles are rather exemplatory and do not describe actual users the author has met.
Their main purpose is to provide a clear dichotomy between two common groups of \acl{SW} users.

\subsection{User} \label{UserUserRole}
\textit{User} has a fairly basic knowledge of using personal computers 
and web-related technologies - knowledge of e.g. \textit{CSS} or \textit{XPath} selectors is expected.
A \textit{User} wants to reach their goal without much additional knowledge and/or specialized tools.

Such a user wants to use the toolkit in the most basic way.
While they might have some experience with the technologies used in the toolkit, they generally do not want to use the toolkit programmatically and rely on the \ac{GUI} tools only.

Their automation use case is easily described, mostly as a linear sequence of well-defined, simple steps.
Some examples of such use cases might be \textit{automated data extraction} and simple \textit{\acl{RPA}}.


\subsection{Developer} \label{DevUserRole}
The \textit{Developer} user role describes an intermediate-to-expert computer specialist with deep
knowledge of computer systems, programming and web-related technologies.
This user role expects to take advantage of the advanced features of the toolkit, possibly spending some extra time learning how to use those properly.

They might not want not only to create and run automations but also to use the toolkit programmatically, 
install the toolkit components on their systems or edit parts of the toolkit. 

When creating an automation, the \textit{Developer} has more complicated use cases with possibly branching scenarios. 
Those might be more \textit{elaborate data extraction} cases, \textit{software testing}, \textit{complicated \ac{RPA}} and other.