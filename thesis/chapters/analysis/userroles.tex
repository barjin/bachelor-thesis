\section{User roles} \label{userroles}

In the first section of the analysis, we describe the typical users of such a toolkit.
Users may have different requirements based on their level of expertise and knowledge.

While the toolkit should be accessible and user-friendly enough to allow beginners to
create web automations with ease; it also should provide the more experienced users with 
an advanced functionality required for handling specific use cases. 
For clarity, we describe only two user roles with a significant difference in skill and knowledge. 

\subsection{False Beginner}
As a \textit{False Beginner}, this user has a fairly basic knowledge of using personal computers 
and web-related technologies - basic knowledge of e.g. \textit{CSS} or \textit{XPath} selectors is expected.
Moreover, such users have a good understanding of their own industry, knowing the domain they want to automate.
A \textit{False Beginner} wants to reach their goal without much additional knowledge and/or specialized tools.

Their automation use case is easily described, mostly as a linear sequence of well-defined, simple steps.
Some examples of such use cases might be \textit{automated data extraction} and simple \textit{\acl{RPA}}.


\subsection{Poweruser}
The \textit{Poweruser} user role describes an intermediate-to-expert computer specialist with deep
knowledge of computer systems, programming and web-related technologies.
This user expects to take advantage of the advanced features of the toolkit, possibly spending some extra time learning how to use those properly.

Such a user has more complicated automation use cases with possibly branching scenarios. 
Those might be more \textit{elaborate data extraction} cases, \textit{software testing}, \textit{complicated \ac{RPA}} and other.