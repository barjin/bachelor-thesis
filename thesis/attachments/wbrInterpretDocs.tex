\section{\texttt{wbr-interpret} code documentation} \label{atta:interpretCode}
This attachment contains in-depth documentation of the \texttt{wbr-interpret} module, with detailed descriptions of the classes and methods
used in the \texttt{wbr-interpret} module.

The following content is structured by individual files.

\subsection{\texttt{interpret.ts}}

The \textit{Interpreter} class defined in the \texttt{interpret.ts} file implements the main logic for the workflow execution.

\subsubsection{Methods}

In the following subsection, both public and private methods of the \texttt{Interpreter} class are described.
The relations between those are also explained here, making it easier to understand the code.

\emptyline
\verb|public async run(page: Page, params?: ParamType): Promise<void>|

\smallskip

This method is the main entry point for the workflow execution.
Using the \textit{Preprocessor}'s methods, it initializes the workflow with new parameters and starts executing the workflow using the \texttt{Interpreter.runLoop()} private method.

This method also registers the \texttt{Interpreter.stopper} callback function for stopping the workflow execution.

\emptyline
\verb|public async stop(): Promise<void>|

\smallskip

This method runs the necessary checks and stops the workflow execution.
In case the checks fail - for example, the interpreter was not running any workflow - this method throws an exception. 

\emptyline
\verb|async runLoop(p: Page, workflow: Workflow): Promise<void>|

\smallskip

This private method represents the main loop of the workflow execution.
Accepting the \textit{Playwright} Page object and an initialized \texttt{Workflow} object as arguments,
it keeps repeatedly looking for the applicable condition among the workflow's conditions and executes the corresponding actions.

It also keeps track of already executed actions and the execution history in general.

\emptyline
\verb|async getState(page:Page, workflow:Workflow):Promise<PageState>|

\smallskip

This private method extracts the state representation from the current browser page context.

Receiving both the Page instance and the currently active workflow, the \texttt{getState} method extracts the smallest representation of the browser's state required 
for the current workflow execution. For example, when extracting the information on elements present in the page, the method first compiles all the selectors from the workflow,
which are then used to query the page for the elements present.

\emptyline
\begin{verbatim}   applicable(where: Where, context: PageState, 
                       usedActions : string[] = []): boolean \end{verbatim}

This private method compares the extracted page state with the conditions of the workflow.

Given a condition from the workflow being executed and the current page state, the \texttt{applicable()} method returns true if the condition is applicable to the current page state.
Optionally it also accepts a list of names of actions that were already executed in the current workflow execution.

\emptyline
\verb|async carryOutSteps(page: Page, steps: What[]) : Promise<void>|

\smallskip

Given a \textit{Page} class instance and a list of actions from the current matched pair, this method carries out the actions on the given Page object.

The implementation of this method also contains the definitions of the custom actions and overrides for some specific \textit{Playwright} methods.

