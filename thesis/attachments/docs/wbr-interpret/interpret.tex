\subsection{\texttt{interpret.ts}}

The \textit{Interpreter} class defined in the \texttt{interpret.ts} file implements the main logic for the workflow execution.

\subsubsection{Type Aliases and Interfaces}

In the following subsection, we define the type aliases and interfaces used in the \texttt{interpret.ts} file.

\emptyline
\verb|interface InterpreterOptions|

\smallskip

This interface describes the object of the optional parameters passed to the \texttt{Interpreter} class constructor.
All of the following fields are optional, as is the whole object.
\begin{itemize}
   \item \texttt{maxRepeats: number} 
   \begin{itemize}
      \item The maximum number of times a condition-action pair can be repeated without interruption before the execution of the workflow is stopped.
      This option defaults to \texttt{5}. Setting this to \texttt{0} disables this loop protection mechanism.
   \end{itemize}
   \item \texttt{maxConcurrency: number}
   \begin{itemize}
      \item The maximum number of concurrent browser tabs that can be handled by the \textit{Interpreter}.
      This option defaults to \texttt{5}. Setting this option to \texttt{null} disables the concurrency limit. 
      This can lead to overloading the browser and thus to a browser crash.
   \end{itemize}
   \item \texttt{serializableCallback: (output: any) => (void | Promise<void>)}
   \begin{itemize}
      \item The function called when a serializable output is received from the browser.
      An object is \textit{Serializable} iff \texttt{JSON.stringify} can be called on it without throwing an exception.
      Defaults to \texttt{console.log}.
   \end{itemize}
   \item \texttt{binaryCallback: (output: any, mimeType: string) => (void | Promise<void>)}
   \begin{itemize}
      \item The function called when a non-serializable output is received from the browser.
      The data is accompanied by its \textit{MIME} type passed in \texttt{mimeType} parameter.
      Default behavior logs a warning message in the console.
   \end{itemize}
   \item \texttt{debug: boolean}
   \begin{itemize}
      \item If set to \texttt{true}, the \textit{Interpreter} will log additional debug messages to the console.
   \end{itemize}
   \item \texttt{debugChannel: Record<string, unknown>}
   \begin{itemize}
      \item An object containing the following fields (all optional):
      \begin{itemize}
         \item \texttt{debugMessage: Function}
         \begin{itemize}
            \item If set, this function is called with all the log messages emitted by the \textit{Interpreter}.
            Can be used to run a remote console for monitoring the automation run.
         \end{itemize}
         \item \texttt{activeId: Function}
         \begin{itemize}
            \item If set, this function is called with every new matched pair's index within the workflow definition.
         \end{itemize}
      \end{itemize}
   \end{itemize}
\end{itemize}



\subsubsection{Methods}

In the following subsection, both public and private methods of the \texttt{Interpreter} class are described.
The relations between those are also explained here, making it easier to understand the code.

\emptyline
\begin{verbatim}
   constructor(workflow:WorkflowFile, 
               options?:Partial<InterpreterOptions>)
\end{verbatim}

The constructor for the Interpreter class.
Accepts a \texttt{WorkflowFile} and a (subset of) \texttt{InterpreterOptions} object,
throws if the \texttt{WorkflowFile} is not a valid workflow definition.

\smallskip

\emptyline
\verb|public async run(page: Page, params?: ParamType): Promise<void>|

\smallskip

This method is the main entry point for the workflow execution.
Using the \textit{Preprocessor}'s methods, it initializes the workflow with new parameters and starts executing the workflow using the \texttt{Interpreter.runLoop()} private method.

This method also registers the \texttt{Interpreter.stopper} callback function for stopping the workflow execution.

\emptyline
\verb|public async stop(): Promise<void>|

\smallskip

This method runs the necessary checks and stops the workflow execution.
In case the checks fail - for example, the interpreter was not running any workflow - this method throws an exception. 

\emptyline
\verb|async runLoop(p: Page, workflow: Workflow): Promise<void>|

\smallskip

This private method represents the main loop of the workflow execution.
Accepting the \textit{Playwright} Page object and an initialized \texttt{Workflow} object as arguments,
it keeps repeatedly looking for the applicable condition among the workflow's conditions and executes the corresponding actions.

It also keeps track of already executed actions and the execution history in general.

\emptyline
\verb|async getState(page:Page, workflow:Workflow):Promise<PageState>|

\smallskip

This private method extracts the state representation from the current browser page context.

Receiving both the Page instance and the currently active workflow, the \texttt{getState} method extracts the smallest representation of the browser's state required 
for the current workflow execution. For example, when extracting the information on elements present in the page, the method first compiles all the selectors from the workflow,
which are then used to query the page for the elements present.

\emptyline
\begin{verbatim}   applicable(where: Where, context: PageState, 
                       usedActions : string[] = []): boolean \end{verbatim}

This private method compares the extracted page state with the conditions of the workflow.

Given a condition from the workflow being executed and the current page state, the \texttt{applicable()} method returns true if the condition is applicable to the current page state.
Optionally it also accepts a list of names of actions that were already executed in the current workflow execution.

\emptyline
\verb|async carryOutSteps(page: Page, steps: What[]) : Promise<void>|

\smallskip

Given a \textit{Page} class instance and a list of actions from the current matched pair, this method carries out the actions on the given Page object.

The implementation of this method also contains the definitions of the custom actions and overrides for some specific \textit{Playwright} methods.

