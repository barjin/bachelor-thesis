\subsection{\texttt{utils/logger.ts}}

The \texttt{logger.ts} file exports the \texttt{logger} function, used for logging messages across the \texttt{wbr-interpret} package.

\subsubsection{Type Aliases and Enums}

\verb|   Level|

\smallskip

The \texttt{Level} enum defines the different log levels.
The number values assigned to the different levels are used for assigning different colours to the messages
and correspond to the ANSI Escape Codes.

\subsubsection{Methods}

\verb$   logger(message: string | Error, level: Level)$

The exported logger function for logging messages.
Prepends every message with a timestamp and sets the message colour based on the value of the \texttt{level} parameter.

\subsection{\texttt{utils/utils.ts}}

A file containing various utility functions used in the \texttt{wbr-interpret} package.

\subsubsection{Methods}

\verb|   arrayToObject(array : any[]) : any|

Converts an array of scalars to an object with items of the array as keys.
All the values are empty arrays.

\subsection{\texttt{utils/concurrency.ts}}

Defines a generic \texttt{Concurrency} class for managing the concurrency of multiple long-running jobs running in parallel.

\subsubsection{Methods}

\verb|   constructor(maxConcurrency: number)|
\smallskip

A constructor for the \texttt{Concurrency} class.
Accepts a \texttt{maxConcurrency} parameter, which defines the maximum number of jobs that can be running in parallel.

\emptyline
\verb|addJob(job: () => Promise<any>) : void |
\smallskip

Passes a job (a time-demanding async function) to the concurrency manager. 
The time of the job's execution depends on the concurrency manager.

The passed function is guaranteed to be called sometime in the future.

\emptyline
\verb|waitForCompletion() : Promise<void>|
\smallskip

Waits until there is no running or a waiting job.
If the concurrency manager is idle at the time of calling this function,
it waits until at least one job is completed.

Returns a \texttt{Promise}, which gets resolved after there is no running or an awaiting job.

\emptyline
\verb|private runNextJob() : void|
\smallskip

A private method taking a waiting job from the queue and processing it. 
Once the job is completed, it calls the \texttt{runNextJob()} method until there are no more jobs in the queue.

\subsection{\texttt{types/}}

The \texttt{types/} directory contains various \acs{TS} typings used in the \texttt{wbr-interpret} package.

\subsection{\texttt{browserSide/scraper.js}}

File containing browser-side code for the \texttt{wbr-interpret} specific actions (\texttt{scrape}, \texttt{scrapeSchema} and other.

\subsubsection{Methods}

\verb|   getBiggestElement(selector)|
\smallskip

Finds the largest element (based on its area) in the DOM tree that matches the given selector.
Currently unused.

\emptyline
\verb|getSelectorStructural(element)|
\smallskip

Returns the structural \acs{CSS} selector of the given element. 
The structural selector describes a path to the element from the root element, based on the tag names.
The returned selector is not unique, as this method is used to find repeating sibling elements on the page in the \texttt{scrapableHeuristics} method.

\emptyline
\verb|scrapableHeuristics(...)|
\smallskip

An method implementing an heuristic for determining the most probably interesting elements on the page.
The ``value'' of the element is determined by amount of similar looking elements on the same page.

\emptyline
\verb|scrape(selector)|
\smallskip

Returns the text content of the elements targetted by \texttt{scrapableHeuristics}.

\emptyline
\verb|scrapeSchema(selector)|
\smallskip

Given a set of element lists, this method matches the elements from the individual lists with each other based on their common ancestors.
This is particularly useful for scraping incomplete tables of elements, where certain columns are not fully populated.