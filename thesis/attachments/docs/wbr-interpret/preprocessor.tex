\subsection{\texttt{proprocessor.ts}}

The \textit{Preprocessor} class defined in the \texttt{preprocessor.ts} file implements the validation and static analysis of the input workflow definitions.

\subsubsection{Methods}

In the following subsection, all the methods of the \texttt{Preprocessor} class are described.
The relations between those are also explained here, making it easier to understand the code.
All of the following methods defined on the class are static, i.e. they can be called without creating an instance of the class.

\emptyline
\verb|validateWorkflow(workflow: WorkflowFile): any|

\smallskip

This method validates the given workflow definition based on the syntax definition from the \autoref{formatDesign} Format Design. 
If the object passed is a valid workflow definition, it returns null.
In case of a syntax error in the workflow definition, it returns an error message describing the problem.

\emptyline
\verb|getParams(workflow: WorkflowFile) : string[]|

\smallskip

For a parametrized workflow, this method returns the list of the parameter names used in the workflow.
Internally, it performs a recursive search for the specific parameter structure.

\emptyline
\verb|extractSelectors(workflow: Workflow) : SelectorArray|

\smallskip

Given a workflow definition, this method extracts the list of the string selectors used in the entire workflow.
This is used for the state matching in the \texttt{Interpreter.runLoop()} method.

\emptyline
\verb|initWorkflow(workflow: Workflow, params?: ParamType) : Workflow|

\smallskip

Initializes the workflow with the given parameters. 
Performs checks on the provided arguments and throws an exception if the parameters are not valid
- the workflow parameters and the provided object do not match, for example.
Also transforms the \verb|{$regex: "regex"}| objects into regular \acs{JS} regular expressions.